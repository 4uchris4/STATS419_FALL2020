\documentclass[]{article}
\usepackage[left=1in,top=1in,right=1in,bottom=1in]{geometry}


%%%% more monte %%%%
% thispagestyle{empty}
% https://stackoverflow.com/questions/2166557/how-to-hide-the-page-number-in-latex-on-first-page-of-a-chapter
\usepackage{color}
% \usepackage[table]{xcolor} % are they using color?

% \definecolor{WSU.crimson}{HTML}{981e32}
% \definecolor{WSU.gray}{HTML}{5e6a71}

% \definecolor{shadecolor}{RGB}{248,248,248}
\definecolor{WSU.crimson}{RGB}{152,30,50} % use http://colors.mshaffer.com to convert from 981e32
\definecolor{WSU.gray}{RGB}{94,106,113}

%%%%%%%%%%%%%%%%%%%%%%%%%%%%

\newcommand*{\authorfont}{\fontfamily{phv}\selectfont}
\usepackage{lmodern}


  \usepackage[T1]{fontenc}
  \usepackage[utf8]{inputenc}




\usepackage{abstract}
\renewcommand{\abstractname}{}    % clear the title
\renewcommand{\absnamepos}{empty} % originally center

\renewenvironment{abstract}
 {{%
    \setlength{\leftmargin}{0mm}
    \setlength{\rightmargin}{\leftmargin}%
  }%
  \relax}
 {\endlist}

\makeatletter
\def\@maketitle{%
  \pagestyle{empty}
  \newpage
%  \null
%  \vskip 2em%
%  \begin{center}%
  \let \footnote \thanks
    {\fontsize{18}{20}\selectfont\raggedright  \setlength{\parindent}{0pt} \@title \par}%
}
%\fi
\makeatother









\title{\textbf{\textcolor{WSU.crimson}{A Tale Of Two
Actors}} \newline \textbf{\textcolor{WSU.gray}{Making A Case For
-actor-}}  }
 

%  

% \author{ \Large true \hfill \normalsize \emph{} }
\author{\Large Christopher
Sarno\vspace{0.05in} \newline\normalsize\emph{Washington State
University}  }


\date{December 13, 2020}
\setcounter{secnumdepth}{2}

\usepackage{titlesec}
% See the link above: KOMA classes are not compatible with titlesec any more. Sorry.
% https://github.com/jbezos/titlesec/issues/11
\titleformat*{\section}{\bfseries}
\titleformat*{\subsection}{\bfseries\itshape}
\titleformat*{\subsubsection}{\itshape}
\titleformat*{\paragraph}{\itshape}
\titleformat*{\subparagraph}{\itshape}

% https://code.usgs.gov/usgs/norock/irvine_k/ip-092225/


%\titleformat*{\section}{\normalsize\bfseries}
%\titleformat*{\subsection}{\normalsize\itshape}
%\titleformat*{\subsubsection}{\normalsize\itshape}
%\titleformat*{\paragraph}{\normalsize\itshape}
%\titleformat*{\subparagraph}{\normalsize\itshape}

% https://tex.stackexchange.com/questions/233866/one-column-multicol-environment#233904
\usepackage{environ}
\NewEnviron{auxmulticols}[1]{%
  \ifnum#1<2\relax% Fewer than 2 columns
    %\vspace{-\baselineskip}% Possible vertical correction
    \BODY
  \else% More than 1 column
    \begin{multicols}{#1}
      \BODY
    \end{multicols}%
  \fi
}





\usepackage{natbib}
\setcitestyle{aysep={}} %% no year, comma just year
% \usepackage[numbers]{natbib}
\bibliographystyle{./biblio/ormsv080.bst}



\usepackage[strings]{underscore} % protect underscores in most circumstances




\newtheorem{hypothesis}{Hypothesis}
\usepackage{setspace}


%%%%%%%%%%%%%%%%%%%%%%%%%%%%%%%%%%%%%%%%%%%%%%%%%%%%%
%%% MONTE ADDS %%%

\usepackage{fancyhdr} % fancy header 
\usepackage{lastpage} % last page 

\usepackage{multicol}


\usepackage{etoolbox}
\AtBeginEnvironment{quote}{\singlespacing\small}
% https://tex.stackexchange.com/questions/325695/how-to-style-blockquote


\usepackage{soul}			%% allows strike-through
\usepackage{url}			%% fixes underscores in urls
\usepackage{csquotes}		%% allows \textquote in references
\usepackage{rotating}		%% allows table and box rotation
\usepackage{caption}		%% customize caption information
\usepackage{booktabs}		%% enhance table/tabular environment
\usepackage{tabularx}		%% width attributes updates tabular
\usepackage{enumerate}		%% special item environment
\usepackage{enumitem}		%% special item environment

\usepackage{lineno}		%% allows linenumbers for editing using \linenumbers
\usepackage{hanging}


\usepackage{mathtools}  	%% also loads amsmath
\usepackage{bm}		%% bold-math
\usepackage{scalerel}	%% scale one element (make one beta bigger font)

\newcommand{\gFrac}[2]{ \genfrac{}{}{0pt}{1}{{#1}}{#2} }

\newcommand{\betaSH}[3]{  \gFrac{\text{\tiny #1}}{{\text{\tiny #2}}}\hat{\beta}_{\text{#3}}   }
\newcommand{\betaSB}[3]{              ^{\text{#1}} _{\text{#2}} \bm{\beta} _{\text{#3}}                   }  %% bold
\newcommand{\bigEQ}{  \scaleobj{1.5}{{\ }= } }
\newcommand{\bigP}[1]{  \scaleobj{1.5}{#1 } }





\usepackage{endnotes}  % he already does this ...
\renewcommand{\enotesize}{\normalsize}
% https://tex.stackexchange.com/questions/99984/endnotes-do-not-be-superscript-and-add-a-space
\renewcommand\makeenmark{\textsuperscript{[\theenmark]}} % in brackets %
% https://tex.stackexchange.com/questions/31574/how-to-control-the-indent-in-endnotes
\patchcmd{\enoteformat}{1.8em}{0pt}{}{}

\patchcmd{\theendnotes}
  {\makeatletter}
  {\makeatletter\renewcommand\makeenmark{\textbf{[\theenmark]} }}
  {}{}



% https://tex.stackexchange.com/questions/141906/configuring-footnote-position-and-spacing

\addtolength{\footnotesep}{5mm} % change to 1mm

\renewcommand{\thefootnote}{\textbf{\arabic{footnote}}}
\let\footnote=\endnote
%\renewcommand*{\theendnote}{\alph{endnote}}
%\renewcommand{\theendnote}{\textbf{\arabic{endnote}}}


\renewcommand*{\notesname}{}

\makeatletter
\def\enoteheading{\section*{\notesname
  \@mkboth{\MakeUppercase{\notesname}}{\MakeUppercase{\notesname}}}%
  \mbox{}\par\vskip-2.3\baselineskip\noindent\rule{.5\textwidth}{0.4pt}\par\vskip\baselineskip}
\makeatother


\renewcommand*{\contentsname}{TABLE OF CONTENTS}

\renewcommand*{\refname}{REFERENCES}


%\usepackage{subfigure}
\usepackage{subcaption}

\captionsetup{labelfont=bf}  % Make Table / Figure bold

%%% you could add elements here ... monte says .... %%%
%\usepackage{mypackageForCapitalH}


%%%%%%%%%%%%%%%%%%%%%%%%%%%%%%%%%%%%%%%%%%%%%%%%%%%%%

% set default figure placement to htbp
\makeatletter
\def\fps@figure{htbp}
\makeatother


% move the hyperref stuff down here, after header-includes, to allow for - \usepackage{hyperref}

\makeatletter
\@ifpackageloaded{hyperref}{}{%
\ifxetex
  \PassOptionsToPackage{hyphens}{url}\usepackage[setpagesize=false, % page size defined by xetex
              unicode=false, % unicode breaks when used with xetex
              xetex]{hyperref}
\else
  \PassOptionsToPackage{hyphens}{url}\usepackage[draft,unicode=true]{hyperref}
\fi
}

\@ifpackageloaded{color}{
    \PassOptionsToPackage{usenames,dvipsnames}{color}
}{%
    \usepackage[usenames,dvipsnames]{color}
}
\makeatother
\hypersetup{breaklinks=true,
            bookmarks=true,
            pdfauthor={Christopher Sarno (Washington State University)},
             pdfkeywords = {},  
            pdftitle={A Tale Of Two Actors: Making A Case For -actor-},
            colorlinks=true,
            citecolor=blue,
            urlcolor=blue,
            linkcolor=magenta,
            pdfborder={0 0 0}}
\urlstyle{same}  % don't use monospace font for urls

% Add an option for endnotes. -----

%
% add tightlist ----------
\providecommand{\tightlist}{%
\setlength{\itemsep}{0pt}\setlength{\parskip}{0pt}}

% add some other packages ----------

% \usepackage{multicol}
% This should regulate where figures float
% See: https://tex.stackexchange.com/questions/2275/keeping-tables-figures-close-to-where-they-are-mentioned
\usepackage[section]{placeins}



\pagestyle{fancy}   
\lhead{\textcolor{WSU.crimson}{\textbf{ A Tale Of Two Actors }}}
\chead{}
\rhead{\textcolor{WSU.gray}{\textbf{  Page\ \thepage\ of\ \protect\pageref{LastPage} }}}
\lfoot{}
\cfoot{}
\rfoot{}


\begin{document}
	
% \pagenumbering{arabic}% resets `page` counter to 1 
%    

% \maketitle

{% \usefont{T1}{pnc}{m}{n}
\setlength{\parindent}{0pt}
\thispagestyle{plain}
{\fontsize{18}{20}\selectfont\raggedright 
\maketitle  % title \par  

}

{
   \vskip 13.5pt\relax \normalsize\fontsize{11}{12} 
   
\textbf{\authorfont Christopher
Sarno} \hskip 15pt \emph{\small Washington State University}   

}

}








\begin{abstract}

    \hbox{\vrule height .2pt width 39.14pc}

    \vskip 8.5pt % \small 

\noindent In this article we compare how \emph{actors} can be quantified
and compared to each other, using \emph{IMDB scores},
\emph{user ratings}, \emph{box office earnings}, and other factors. The
goal is to determine which actor is better between Will Smith and Denzel
Washington. \vspace{0.25in}


    



    
    \hbox{\vrule height .2pt width 39.14pc}
    \vskip 5pt 
    \hfill \textbf{\textcolor{WSU.gray}{ December 13, 2020 } }
    \vskip 5pt 
    
\end{abstract}


\vskip -8.5pt



 % removetitleabstract

\noindent  

\section{Introduction}
\label{sec:intro}

Movies are an art form, and as is such with all art, is always
subjective as to which is the best. The same logic can be said of actors
in those movies. A phenomenal actor could be stifled by tired and lazy
writing just the same as a weak actor could be carried by the story and
production value that surrounds him.

\vspace{2.5mm}

While quantifying and comparing actors is difficult, it is not
impossible. Throughout this report, I will attempt to convince in favor
of -actor- using IMDB records, user ratings, etc. Objectivity is crucial
here, so both sides of the argument will be given and commented upon. It
should be noted that this report takes only into account quantifiable
and comparable statistics, not range of emotion, character depth or any
other factors based on subjective opinions and feelings.

\newpage

See Figure \ref{fig:summary}.

\section{Summary Table:  Through analysis of data scraped from the IMDB website, I have compiled a table of meaningful statistics by which to compare Will Smith and Denzel Washington. }
\label{sec:rq}

\begin{figure}[!ht]
%% figures have hrule, tables have hline
    \hrule
    \caption{ \textbf{Summary Table} }
    \begin{center}
        \scalebox{1.00}{    \includegraphics[trim = 0 0 0 0,clip,width=\textwidth]{figures/will-denzel-summary.pdf} }
    \end{center}
    \label{fig:summary}
  \hrule
  \vspace{2.5mm}
      \caption{\textbf{ A compilation of comparable stats for Will and Denzel }   }
      \label{fig:compilation}
  \vspace{-2.5mm}
  \hrule
\end{figure}
\newpage

\newpage

\subsection{Commentary}
\label{sec:summary-commentary}

As could be inferred from the beginning, no actor has a clear advantage
at first glance. Some statistics here indicate a race that ranges from
extremely close to dead tied. The standout comparisons here are mean and
max box office earnings, and mean and max movie ratings from Metacritic.
Will Smith takes a large lead in average and max box office earnings,
with the average almost doubling that of Denzel Washington (\$103.42
million vs \$52.12 million) and the max almost tripling that of Denzel
(\$355.56 million vs \$130.16 million).

Denzel's stats are no slouch either, however. He overtakes Will Smith by
small but noticeable margins in fan movie ratings, including a
devastating lead in minimum movie rating. His Metacritic ratings are
where he really starts to shine. Denzel Washington's average and maximum
movie ratings from Metacritic show significant leads over Will Smith's,
with a lead of 10.28 point lead in average and a 7 point lead in maximum
movie rating.

\newpage

\begin{figure}[!ht]
%% figures have hrule, tables have hline
    \hrule
    \caption{ \textbf{Box Office Earnings} }
    \begin{center}
        \scalebox{1.00}{    \includegraphics[trim = 0 0 0 0,clip,width=\textwidth]{figures/will-denzel-box-office.pdf} }
    \end{center}
    \label{fig:box-office}
  \hrule
  \vspace{2.5mm}
      \caption{\textbf{ Plotting box office earnings by year }   }
      \label{fig:combined}
  \vspace{-2.5mm}
  \hrule
\end{figure}

\newpage

\section{Key Findings}
\label{sec:findings}

Looking strictly at the table in Figure \ref{fig:summary}, a case could
be made for either actor. However, I believe Will Smith is the clear
victor here. Money speaks louder than words, and Will Smith's box office
earnings absolutely blow Denzel Washington's out of the water. Denzel
Washington's leads in other areas can be discredited by the fact that
Will Smith's worst movies are outliers that bring down his averages in
movie ratings from both fans and Metacritic. It's clear from the numbers
that despite a few flops, Will Smith's movies are far more popular than
Denzel Washington's, and it shows from peoples' willingness to spend far
more money on his movies than Denzel's. This lead is put into a better
visual perspective in Figure \ref{fig:box-office}.

\section{Conclusion}
\label{sec:conclusion}

To summarize, despite Denzel's small leads in certain Metacritic and fan
rating metrics, Will Smith is the better actor of the two. In the areas
which he outshines Denzel, he does so by a much larger factor than in
the areas where he is outperformed. Even in a world where fan and
Metacritic ratings are the only factor that determine an actor's
performance and aptitude, Denzel would still barely come out on top due
to a small number of extremely poor performing movies from Will Smith
that brings down his averages.

\vspace{0.5in}

Once again, nobody defeats Agent J. \newpage

\vspace{0.5in}

\newpage




%% appendices go here!


\newpage
\theendnotes

%%%%%%%%%%%%%%%%%%%%%%%%%%%%%%%%%%%  biblio %%%%%%%%
\newpage
\begin{auxmulticols}{1}
\singlespacing 
\bibliography{./biblio/master.bib}

%%%%%%%%%%%%%%%%%%%%%%%%%%%%%%%%%%%  biblio %%%%%%%%
\end{auxmulticols}

\newpage
{
\hypersetup{linkcolor=black}
\setcounter{tocdepth}{3}
\tableofcontents
}



\end{document}